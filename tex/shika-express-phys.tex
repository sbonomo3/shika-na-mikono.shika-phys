\setcounter{secnumdepth}{0}

\chapter{Shika Express Demonstrations for Physics}

\section{Egg Float}

\begin{description*}
\item[Topic:]{Density/Relative Density (Form 1)}
\item[Materials:]{2 fresh eggs, 2 containers (bottles cut in half), salt (less than half a cup)}
\item[Setup:]{Fill two containers with water and place a fresh egg in each. }
\item[Procedure:]{Leave one as it is and add salt to the other. Add and mix salt until the egg floats in the saltwater container.}
%\item[Hazards:]{}
\item[Questions:]{Why does the egg float in saltwater but sink in fresh water?}
\item[Theory:]{Saltwater has a greater density than fresh water. A fresh egg has a density between fresh water and saltwater. Since an egg is denser than freshwater, it sinks. Since an egg is less dense than saltwater, it floats.}
\item[Applications:]{This is the same reason why it is easier to stay afloat when swimming in the ocean (saltwater) as opposed to a lake (fresh water).}
%\item[Notes:]{}
\end{description*}

\begin{center}
\includegraphics[width=5cm]{./img/egg-float.png}
\end{center}


\section{Pressure in a Bottle}

\begin{description*}
\item[Topic:]{Pressure (Form 1)}
\item[Materials:]{1.5 L bottle, syringe needle or pin/nail, water}
\item[Setup:]{Use a syringe needle, heated nail, etc. to poke three holes into a bottle. Put one hole near
the bottom, one near the middle, and the last hole between them.}
\item[Procedure:]{Fill the bottle with water and place on a table. Observe the trajectories of water coming from the three holes.}
%\item[Hazards:]{}
\item[Questions:]{}\hfill
\begin{enumerate*}
\item What do you notice about the trajectory of the water from each hole? Which one reaches the furthest horizontal distance?
\item How does the pressure change with the depth of the water? Why?
\end{enumerate*}
\item[Theory:]{The water flowing from the lower holes follows a shallower arc and hits the ground further from the bottle. The added weight of the water above the lower holes increases the pressure, resulting in an increased initial horizontal velocity. The relationship that pressure increases with depth is shown. ($P = \rho g h$)}
\item[Applications:]{The wall of a dam is made much thicker at the bottom than at the top. Also, water storage tanks are placed at the top of a building. This is because the pressure in a liquid is related to its depth.}
\item[Notes:]{There is a difference between depth and height. Height is measured from the reference point upward while depth is measured from the reference point downward. The reference point in this case is at the surface of water.}
\end{description*}

\begin{center}
\includegraphics[width=8cm]{./img/pressure-liquid-alt.png}
\end{center}

%\begin{figure}
%\begin{center}
%\def\svgwidth{150pt}
%\input{./img/pressure-liquid.pdf_tex}
%%\caption{Demonstration of the effect of depth on liquid pressure}
%\label{fig:pressure-liquid}
%\end{center}
%\end{figure}




%\end{multicols}

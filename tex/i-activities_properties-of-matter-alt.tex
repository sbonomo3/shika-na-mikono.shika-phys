\section{Structure and Properties of Matter}

\begin{multicols}{2}

\section*{States of Matter} \label{sec:states-matter}
All matter is made up of particles. These particles are in constant motion which increases with their temperature. Depending on temperature, matter may exist in three states: \emph{solid}, \emph{liquid} or \emph{gas}.


\subsection{Student Particles}

\begin{center}
\includegraphics[width=0.4\textwidth]{./img/source/states-matter.png}
%\includegraphics[width=0.4\textwidth]{./img/vso/states-matter.png}
\end{center}

\begin{description*}
%\item[Subtopic:]{}
%\item[Materials:]{}
%\item[Setup:]{}
\item[Procedure:]{Use students to demonstrate the concept of states of matter.}
%\item[Hazards:]{}
%\item[Questions:]{}
%\item[Observations:]{}
\item[Theory:]{When students or objects are close together, they represent particles in the \emph{solid} state. As they move apart and past each other they represent particles in the \emph{liquid} state. Fast and randomly moving pupils or objects represent particles in the \emph{gaseous} state.}
%\item[Applications:]{}
%\item[Notes:]{}
\end{description*}

\subsection{A Model of Motion}

\begin{center}
\includegraphics[width=0.49\textwidth]{./img/vso/motion-model.png}
\end{center}

\begin{description*}
%\item[Subtopic:]{}
%\item[Materials:]{}
%\item[Setup:]{}
\item[Procedure:]{Put some dry beans, rice or stones in a clear bottle. Hold the bottle still, then turn it, then shake it vigorously.}
%\item[Hazards:]{}
\item[Questions:]{Which activity corresponds to which state of matter?}
%\item[Observations:]{}
\item[Theory:]{The movement of particles in solids is small and hence they are in fixed order. In liquids the particles move past each other and have lost the stiff order. In gases they move very fast and randomly, losing all order.}
%\item[Applications:]{}
%\item[Notes:]{}
\end{description*}

\subsection{Changes of State}

\begin{center}
\includegraphics[width=0.25\textwidth]{./img/source/change-state.png}
\end{center}

\begin{description*}
%\item[Subtopic:]{}
\item[Materials:]{Tin can, glass bottle, water, \nameref{sec:heatsources}*}
%\item[Setup:]{}
\item[Procedure:]{Pour a small amount of water into a tin can and heat it until it boils. Fill a bottle with cool water and hold it above the tin can.}
%\item[Hazards:]{}
%\item[Questions:]{}
\item[Observations:]{Water drops form on the outside of the cool bottle when it is touched by the steam of the boiling water.}
\item[Theory:]{Water particles escape from the boiling water as vapour and condense on the lower surface of the bottle to form water droplets. This is indirect evidence that water is made up of small particles.}
%\item[Applications:]{}
%\item[Notes:]{}
\end{description*}

%==================================================================================================%

\section*{Particulate Nature of Matter}


\subsection{Salt is Made of Particles}

\begin{center}
\includegraphics[width=0.4\textwidth]{./img/source/salt-particles.png}
\end{center}

\begin{description*}
%\item[Subtopic:]{}
\item[Materials:]{Salt/sugar, cup, water}
%\item[Setup:]{}
\item[Procedure:]{Roll some salt or sugar crystals between your fingers to feel their hardness. Taste the crystals. Take a sip of the water. Now put the salt or sugar crystals in the water and shake it. Taste again.}
%\item[Hazards:]{}
%\item[Questions:]{}
\item[Observations:]{The crystals are hard and of cubical shape. They dissolve in water and the solution tastes like salt or sugar.}
\item[Theory:]{Sugar and salt are made up of tiny particles that can be identified by tasting even though they can not be seen as a solution.}
%\item[Applications:]{}
%\item[Notes:]{}
\end{description*}

\subsection{Weighing Particles}

\begin{center}
\includegraphics[width=0.4\textwidth]{./img/source/weighing-particles.png}
\end{center}

\begin{description*}
%\item[Subtopic:]{}
\item[Materials:]{\nameref{sub:beambalance}*, small pieces of wood, \nameref{sec:heatsources}}
%\item[Setup:]{}
\item[Procedure:]{Weight pieces of wood and record the weight. Then burn the wood and weigh the ash.}
%\item[Hazards:]{}
\item[Questions:]{Is there a difference between the two weights?}
%\item[Observations:]{}
\item[Theory:]{The weight of the ash is less than that of wood. The loss in weight is due to particles which escaped as soot and gas.}
\item[Applications:]{This is why garbage reduces in size when burned. Burning wood and garbage releases carbon dioxide and other harmful gases into our environment. This is one form of \emph{pollution}.}
%\item[Notes:]{}
\end{description*}

%==================================================================================================%

\section*{Elasticity}


\subsection{Hooke's Law}
\textbf{*NECTA PRACTICAL*}
\begin{center}
\includegraphics[width=0.49\textwidth]{./img/source/elasticity.png}
\end{center}

\begin{description*}
%\item[Subtopic:]{}
\item[Materials:]{Rubber band/elastic strip, ruler, staple pin, \nameref{sec:retort-stand}*, \nameref{sub:scalepans}*, nails/small \nameref{sec:masses}*}
\item[Setup:]{Fix a rubber band at one end to a table or retort stand. At the other end, attach a paper clip to act as a pointer and a small bag or scale pan.}
\item[Procedure:]{Fill the scale pan with successive numbers of nails or known weights. Have students measure the extension of the rubber band each time they add more weights. Record the readings and use the data to draw a graph of force (weight) against extension.}
%\item[Hazards:]{}
\item[Questions:]{What is the relationship between number of weights added and extension of the rubber band? What does the slope of the graph represent?}
%\item[Observations:]{}
\item[Theory:]{Hooke's Law states that the force applied to an elastic object is directly proportional to its extension ($F = kx$). Graphing the applied \emph{weight} (in Newtons) against the extension (in metres), the slope represents the spring constant of the elastic material in N/m. }
%\item[Applications:]{}
%\item[Notes:]{}
\end{description*}

%==================================================================================================%

\section*{Adhesion and Cohesion}
Forces between particles of the same material are called \emph{cohesive forces} while those between particles of different materials are called \emph{adhesive forces}.


\subsection{Exploring Adhesion and Cohesion}

\begin{center}
\includegraphics[width=0.4\textwidth]{./img/adhesion-cohesion.png}
\end{center}

\begin{description*}
%\item[Subtopic:]{}
\item[Materials:]{Sheet of glass, water, honey, glycerin, cooking oil, syringe, and 2 wooden blocks}
%\item[Setup:]{}
\item[Procedure:]{Place a sheet of glass over two wooden blocks on a table. Using a syringe, place a drop of different liquids on the glass.}
%\item[Hazards:]{}
%\item[Questions:]{}
\item[Observations:]{Water spreads and wets the glass, while honey, glycerin and cooking oil remain in a spherical shape.}
\item[Theory:]{The adhesive forces between the water molecules and glass molecules are greater, while the cohesive forces between the molecules of honey, glycerin and cooking oil are larger.}
%\item[Applications:]{}
%\item[Notes:]{}
\end{description*}

\subsection{Pinching Water}
\begin{description*}
%\item[Subtopic:]{}
\item[Materials:]{500 mL water bottle, needle/pin/small nail}
\item[Setup:]{Make 5 small holes at the bottom of the bottle with a syringe needle or nail. Make them close together (about 5 mm apart).}
\item[Procedure:]{Fill the bottle with water and allow it to flow through the holes at the bottom. Use your thumb and forefinger to pinch the streams together to form a single stream. Pass your hand in front of the holes and all five will appear again.}
%\item[Hazards:]{}
%\item[Questions:]{}
%\item[Observations:]{}
\item[Theory:]{Water has a tendency to cling to itself due to its surface tension and cohesion. As you bring the streams together, you allow the water to stick to itself forming a single stream. Passing your hand in front again stops the flow of water at the holes and allows it to start again, which it will do in five streams.}
%\item[Applications:]{}
%\item[Notes:]{}
\end{description*}

\subsection{Water Drops}

\begin{center}
\includegraphics[width=0.3\textwidth]{./img/source/water-drops.png}
\end{center}

\begin{description*}
%\item[Subtopic:]{}
\item[Materials:]{Syringe or water dropper}
%\item[Setup:]{}
\item[Procedure:]{Slowly drip water from the syringe or water dropper. Observe how the drop forms.}
%\item[Hazards:]{}
%\item[Questions:]{}
\item[Observations:]{The water stream grows thinner and thinner as it moves further down and finally breaks to form drops.}
\item[Theory:]{Strong cohesive forces hold the water molecules together, until they are overcome by gravity and the water breaks off as drops.}
%\item[Applications:]{}
%\item[Notes:]{}
\end{description*}

%==================================================================================================%

\section*{Surface Tension}


\subsection{Pin Float}

\begin{center}
\includegraphics[width=0.49\textwidth]{./img/source/pin-float.png}
\end{center}

\begin{description*}
%\item[Subtopic:]{}
\item[Materials:]{Cup or small dish, straight pin/razor/paper clip, water, detergent}
%\item[Setup:]{}
\item[Procedure:]{Fill the cup with clean water and carefully float a pin, razor or small paper clip. Now add a small amount of detergent to the water and observe what happens.}
%\item[Hazards:]{}
%\item[Questions:]{}
\item[Observations:]{The objects float on the surface of the water initially, but after adding detergent, they sink to the bottom.}
\item[Theory:]{The surface tension of the water acts as an elastic membrane and is strong enough to support the small objects. Soap lowers the surface tension of water and therefore the objects sink.}
%\item[Applications:]{}
%\item[Notes:]{}
\end{description*}

\subsection{Water Dome}

\begin{center}
\includegraphics[width=0.2\textwidth]{./img/source/water-dome.png}
\end{center}

\begin{description*}
%\item[Subtopic:]{}
\item[Materials:]{Coin, water, syringe or eyedropper}
%\item[Setup:]{}
\item[Procedure:]{Place the coin flat on a table. Use the syringe or eyedropper to carefully drop individual water drops onto the coin.}
%\item[Hazards:]{}
\item[Questions:]{How many drops do you think the coin can hold?}
\item[Observations:]{The coin holds a surprising number of drops and forms a dome shape before the water spills over.}
\item[Theory:]{The surface tension of the water holds it together against the force of gravity, which is trying to pull the water off the coin.}
%\item[Applications:]{}
%\item[Notes:]{}
\end{description*}

\subsection{Blowing Bubbles}
\begin{description*}
%\item[Subtopic:]{}
\item[Materials:]{Thin piece of wire (approximately 30cm), water, detergent, glycerin (optional)}
\item[Setup:]{Bend the wire to form a loop of 2 to 3 cm in diameter, circling this loop many times. Leave a straight piece several cm long as a handle. Make a concentrated solution of detergent in water with a small amount of glycerin.}
\item[Procedure:]{Dip the circular part of the wire into the detergent. You should see a thin soapy film across the circle upon removal. Gently blow through the circle until a bubble separates from the wire.}
%\item[Hazards:]{}
%\item[Questions:]{}
\item[Observations:]{While blowing, the solution is being pulled back towards the surface. Once it breaks free as a bubble, it forms a spherical shape.}
\item[Theory:]{The surface tension of water causes the bubble to form the shape with the minimum surface area, which is a sphere.}
%\item[Applications:]{}
%\item[Notes:]{}
\end{description*}

%==================================================================================================%

\section*{Capillarity}

\subsection{Capillary Rise}
\begin{description*}
%\item[Subtopic:]{}
\item[Materials:]{Clear thin plastic straws with different diameters, shallow container (bottom of a water bottle/jar cap), various liquids, e.g. water, spirit and cooking oil}
%\item[Setup:]{}
\item[Procedure:]{Place one end of a straw into a container of water 1 cm deep so that the end is submerged but not touching the bottom. Mark the change in water level in the straw after about a minute. Repeat for different liquids and different size straws.}
%\item[Hazards:]{}
\item[Questions:]{Which liquid rises the farthest up the straw? Do liquids rise faster in wide or thin straws?}
\item[Observations:]{The spirit rises to the greatest height while water rises the least. Liquids rise faster in thin straws compared to thick ones.}
\item[Theory:]{Capillary rise results from adhesion, allowing the liquid to climb along the surface of the tube, as well as cohesion, which pulls the remainder of the liquid up. In a thin container, a larger proportion of liquid is attached to the side of the tube and a smaller proportion is being held by surface tension, so the adhesive force is strong enough to pull all the liquid up the tube.}
%\item[Applications:]{}
%\item[Notes:]{}
\end{description*}

\subsection{Moving Matches}

%\begin{center}
%\includegraphics[width=0.4\textwidth]{./img/.png}
%\end{center}

\begin{description*}
%\item[Subtopic:]{}
\item[Materials:]{Matches, water, straw, plastic lid}
%\item[Setup:]{}
\item[Procedure:]{Break several matches near the middle, but not so that they come apart. They should make acute angles. Place them on the plastic lid and place a few drops of water on the broken joints of the matches using the straw.}
%\item[Hazards:]{}
%\item[Questions:]{}
\item[Observations:]{The matches close and return to their original straight shape.}
\item[Theory:]{Water gets absorbed in the wooden matchstick and causes it to expand.}
\item[Applications:]{This is why it is difficult to open a wooden door after it rains. The water rises up the wood causing it to expand into its frame.}
%\item[Notes:]{}
\end{description*}

\subsection{Automatic Irrigation}

\begin{center}
\includegraphics[width=0.3\textwidth]{./img/source/irrigation.png}
\end{center}

Capillary action can be used to provide automatic irrigation for plants. Students can perform irrigation by dipping a porous material such as paper or cotton cloth in water.

%==================================================================================================%

\section*{Diffusion}

\subsection{Diffusion in Liquids}

\begin{center}
\includegraphics[width=0.4\textwidth]{./img/vso/diffusion.png}
\end{center}

\begin{description*}
%\item[Subtopic:]{}
\item[Materials:]{Plastic water bottle, food colour (liquid or powder)}
%\item[Setup:]{}
\item[Procedure:]{Put a drop or small amount of powdered food colour into the water without shaking and observe what happens.}
%\item[Hazards:]{}
%\item[Questions:]{}
\item[Observations:]{The colour gradually spreads throughout the water.}
\item[Theory:]{This spreading is due to the motion of the particles of food colour. This process is called \emph{diffusion}.}
\item[Applications:]{Organisms utilize diffusion to balance nutrient concentrations in cells and to transfer oxygen into the bloodstream during respiration.}
%\item[Notes:]{}
\end{description*}

\subsection{Smelling Particles}

\begin{description*}
%\item[Subtopic:]{}
\item[Materials:]{Orange or other citrus fruit, box}
%\item[Setup:]{}
\item[Procedure:]{Peel and orange and have students raise their hands when they begin to smell it. Now place a box in front of the orange and repeat the test.}
%\item[Hazards:]{}
%\item[Questions:]{}
\item[Observations:]{Students in the front center of the room should be the first to raise their hands, followed by those near the sides and in the back. When the orange is peeled behind the box it takes longer for the smell to reach the students.}
\item[Theory:]{Tiny particles from the orange peel spread by diffusion to students' noses. The box hinders the motion of the particles and so they reach the students more slowly.}
\item[Applications:]{Air fresheners and other sprays}
%\item[Notes:]{}
\end{description*}

%==================================================================================================%

\section*{Osmosis}


\subsection{Vanilla Balloon}

\begin{center}
\includegraphics[width=0.3\textwidth]{./img/vso/osmosis-vanilla.png}
\end{center}

\begin{description*}
%\item[Subtopic:]{}
\item[Materials:]{Balloon/plastic bag, vanilla, straw/syringe}
%\item[Setup:]{}
\item[Procedure:]{Place a few drops of vanilla in a deflated balloon. Now blow up the balloon and tie it shut.}
%\item[Hazards:]{}
%\item[Questions:]{}
\item[Observations:]{You can smell the vanilla through the surface of the balloon.}
\item[Theory:]{The balloon acts as a \emph{semi-permeable membrane} which allows some of the vanilla particles to pass through and reach your nose. Other particles remain inside the balloon.}
%\item[Applications:]{}
%\item[Notes:]{}
\end{description*}

\subsection{Semi-Permeable Membranes}

\begin{center}
\includegraphics[width=0.4\textwidth]{./img/vso/membrane.png}
\end{center}

\begin{description*}
%\item[Subtopic:]{}
\item[Materials:]{Glass jar, clear plastic bag, small beads or stones, beans, netting, string/rubber band}
\item[Setup:]{Place the mixture of beads and beans in the jar. Place the net and plastic bag over the top and tie them on securely.}
\item[Procedure:]{Shake the apparatus for a few seconds.}
%\item[Hazards:]{}
%\item[Questions:]{}
\item[Observations:]{Only the small beads pass through the netting. The beans remain in the jar.}
\item[Theory:]{The beads represent small molecules and the net is a semi-permeable membrane. The beans are too large to pass through and hence remain in the jar.}
\item[Applications:]{Water filters, organism cell membranes}
%\item[Notes:]{}
\end{description*}


\subsection{Potato Osmosis}

\begin{center}
\includegraphics[width=0.49\textwidth]{./img/vso/osmosis-potato-full.png}
\end{center}

\begin{description*}
%\item[Subtopic:]{}
\item[Materials:]{Potato, 2 water bottles, salt, water}
\item[Setup:]{Cut two equal size pieces of potato. Fill one bottle with fresh water and the other with a salt water solution.}
\item[Procedure:]{Put one piece of potato in each bottle. Observe over the next few hours.}
%\item[Hazards:]{}
%\item[Questions:]{}
\item[Observations:]{The potato in fresh water swells while the potato in salt water shrivels up.}
\item[Theory:]{Through osmosis, water moves from a region of low concentration to one of high concentration through a semi-permeable membrane (the potato). In fresh water, the potato has the higher salt concentration, so water enters in order to make a balance. In salt water, the concentration of the surrounding water is higher than that of the potato, so water inside the potato moves outside to dilute the salt solution.}
%\item[Applications:]{}
\item[Notes:]{Try this experiment again with a boiled potato. Do you observe any differences?}
\end{description*}


\end{multicols}

\pagebreak